\subsection{Measurements Representation}
Scan measurements $\mathbf{s}_i=[r_i,\theta_i,\psi_i]^\text{T}$ are transformed
into their corresponding Cartesian 3D coordinates $\mathbf{p}_i=[x_i,y_i,z_i]
^\text{T}$, where $r_i$ is a range measurement, $\theta_i$ a pitch angle, and
$\psi_i$ a bearing angle. From a complete laser sweep, we obtain a point cloud
representation $\mathcal{P}=\{\mathbf{p}_1,\mathbf{p}_2,\dots,\mathbf{p}_N\}$.
$\mathcal{P}$ is finally projected onto a 2D grid $\mathcal{G}=\{\mathcal{C}_1,
\mathcal{C}_2,\dots,\mathcal{C}_M\}$, with cells $\mathcal{C}_i=
\{\mathbf{ul}_i,\mathbf{lr}_i,\mathbf{c}_i,h_i,l_i,\mathcal{I}_i\}$, where
$\mathbf{ul}_i$ and $\mathbf{lr}_i$ are the upper left and lower right 2D
coordinates of the cell, $\mathbf{c}_i$ the center of the cell, $h_i$ the height
of the cell such that $h_i\propto\mathcal{N}(\mu_i,\sigma_i)=p(h_i\mid\mu_i,
\sigma_i,\mathcal{I}_i)$, $l_i\in\{1,\dots,M\}$ a discrete label for the cell,
and $\mathcal{I}_i=\{\mathbf{p}_j\mid\mathbf{p}_j\in\mathcal{P},
\mathbf{p}_{j_x}>\mathbf{ul}_{i_x},\mathbf{p}_{j_y}>\mathbf{ul}_{i_y}
,\mathbf{p}_{j_x}<\mathbf{lr}_{i_x},\mathbf{p}_{j_y}<\mathbf{lr}_{i_y}
,\mathbf{p}_{j_z}<\theta_U,\mathbf{p}_{j_z}>\theta_L\}$. $\theta_U$
and $\theta_L$ define upper and lower bounds for the height values.

The grid $\mathcal{G}$ will also be referred to as a Digital Elevation Map (DEM)
in the rest of the paper. The choice of a DEM representation is mainly guided by
the final outcome of the algorithm, i.e. a traversability map for the planning
process. It is also convenient for defining Regions of Interest (ROI) in
$\mathcal{P}$ and for simplifying the subsequent computations. We could also
have resorted to a polar grid representation, so as to have a finer resolution
close to the sensor (THIS IS STILL OPEN AND NEED FURTHER EXPLANATIONS). Whenever
the number of points in a cell $\mathcal{C}_i$ is below a threshold, i.e.
$|\mathcal{I}_i|<\theta_P$, it is flagged as invalid.

REMARKS:
\begin{itemize}
\item EXPLAIN HOW WE CAN GO FROM THE PUSH-BROOM LASER DATA TO THE DEM
\item EXPLAIN WHAT HEIGHT $h_i$ IS FINALLY TAKEN
\end{itemize}

\subsection{Environment Model and Inference Task}
We assume a piecewise planar environment, i.e., the observed scene is composed
of a set of plane segments $\mathcal{S}_1,\mathcal{S}_2,\dots,\mathcal{S}_M$, with
$\mathcal{S}_i=\{\mathcal{C}_j\mid h_j=\mathbf{w}_i^\text{T}\boldsymbol{\phi}
(\mathbf{c}_{j})\}$, $\mathbf{w}_i=[w_0^{(i)}, w_1^{(i)}, w_2^{(i)}]^\text{T}$,
and $\boldsymbol{\phi}(\mathbf{c}_{j})=[1,\mathbf{c}_j]^\text{T}$.

Boundaries between plane segments define local heights discontinuities that we
shall term \emph{curbs} from now on. The major inference task therefore boils
down to discovering those plane segments. To this end, we follow a probabilistic
and iterative approach.
